%
% comp2.tex Complexity section - Properties
%
% This is part of the common section of the complexity section -
% used in both HTML and PS versions
%

\subsection{Properties of the \Ohx\ notation}
The following general properties of \Ohx\ notation expressions may
be derived:
\begin{enumerate}
\item Constant factors may be ignored:

For all $k > 0$, {\bf kf} is \Oh{f}.

\EG $a n^2$ and $b n^2$ are both \Oh{n^2}.


\item Higher powers of \nx\ grow faster than lower powers:
\label{higher}

$n^r$ is \Oh{n^s} if $0 \leq r \leq s$.

\item The growth rate of a sum of terms is the growth rate of its
fastest growing term:
\label{sum}

If \varx{f} is \Oh{g}, then \varx{f + g} is \Oh{g}.

\EG $an^3 + bn^2$ is \Oh{n^3}.

\item The growth rate of a polynomial is given by the growth
rate of its leading term (\CF (2), (3)):

If \varx{f} is a polynomial of degree \varx{d},
then \varx{f} is \Oh{n^d}.

\item If \varx{f} grows faster than \varx{g}, which grows faster
than \varx{h}, then \varx{f} grows faster than \varx{h}.

\item The product of upper bounds of functions gives an upper
bound for the product of the functions:

If \varx{f} is \Oh{g} and \varx{h} is \Oh{r}, then
$fh$ is \Oh{gr}

\EG if \varx{f} is \Oh{n^2} and \varx{g} is \Oh{\log{n}},
then \varx{fg} is \Oh{n^2\log{n}}.

\item Exponential functions grow faster than powers:

$n^k$ is \Oh{b^n}, for all $b > 1, k \geq 0$,

\EG $n^4$ is \Oh{2^n} and
$n^4$ is \Oh{exp(n)}.

\item Logarithms grow more slowly than powers:

$\log_b{n}$ is \Oh{n^k} for all $b > 1, k > 0$

\EG $\log_2{n}$ is \Oh{n^{0.5}}.


\item All logarithms grow at the same rate: 

$\log_b{n}$ is $\Theta(\log_d{n})$ for all $b, d > 1$.

\item The sum of the first \nx\ $r^{th}$ powers grows as
the $(r+1)^{th}$ power:

\[ \sum_{k=1}^{n} k^{r} is \Theta(n^{r+1})  \]
\begin{center}
\EG \( \sum_{k=1}^{n} i = \frac{(n+1)n}{2} is \Theta(n^2))  \)
\end{center}

\end{enumerate}

\subsection{Polynomial and Intractable Algorithms}

\subsubsection{Polynomial time complexity}

An algorithm is said to have {\it polynomial time complexity}
{\it iff} it is \Oh{n^d} for some integer \varx{d}.

\subsubsection{Intractable Algorithms}
A problem is said to be {\it intractable} if no algorithm 
with polynomial time complexity is known for it.
We will briefly examine some intractable problems in
a later section.

