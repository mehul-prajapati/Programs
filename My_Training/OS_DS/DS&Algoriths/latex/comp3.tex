%
% comp3.tex Complexity section - Analysis
%
% This is part of the common section of the complexity section -
% used in both HTML and PS versions
%

\subsection{Analysing an algorithm}

\subsubsection{Simple Statement Sequence}
First note that a sequence of statements which is executed
once only is \Oh{1}.
It doesn't matter how many statements are in the sequence -
only that the number of statements (or the time that they
take to execute) is constant for all problems.

\subsubsection{Simple Loops}
If a problem of size \nx\ can be solved with a simple loop: \\

\begin{Ccode}
for(i=0;i<n;i++)\\
\>	\{ s; \}\\
\end{Ccode}

where s is an \Oh{1}\ sequence of statements, then the
time complexity is \nx\Oh{1}\ or \Oh{n}.

If we have two nested loops:

\begin{Ccode}
for(j=0;j<n;j++)\\
\>    for(i=0;i<n;i++)\\
\>\>        \{ s; \}\\
\end{Ccode}

then we have \nx\ repetitions of an \Oh{n}\ sequence,
giving a complexity of: \nx\Oh{n}\ or \Oh{n^2}.

Where the index 'jumps' by an increasing amount in each iteration,
we might have a loop like: 

\begin{Ccode}
h = 1;\\
while( h $\leq$ n )\\
\>    \{ s;\\
\>    h = 2*h; \}
\end{Ccode} 

in which {\tt h} takes values 1, 2, 4, ... until it exceeds \nx.
This sequence has $1 + \lfloor{\log_2{n}\rfloor }$ values, so the 
complexity is \Oh{\log_2{n}}.

If the inner loop depends on an outer loop index:
\begin{verbatim}
for(j=0;j<n;j++)
    for(i=0;i<j;i++)
        { s; }
\end{verbatim}
The inner loop {\tt for(i=0; .. } gets executed \varx{i}\ times, 
so the total is:
\[ \sum_1^n{i} = \frac{n(n+1))}{2} \]
and the complexity is \Oh{n^2}.
We see that this is the same as the result for two nested 
loops above,
so the variable number of iterations of the inner loop 
doesn't affect the `big picture'.

However, if the number of iterations of one of the loops
decreases by a constant factor with every iteration:
\begin{verbatim}
h = n;
for(j=0;j<h;j++)
    {
    for(i=0;i<n;i++)
        { s; }
    h = h/2;
    }
\end{verbatim}
Then:
There are $\log_2{n}$\ iterations of the outer loop and the inner loop
is \Oh{n}, so the overall complexity is
\Oh{{n\log{n}}}.
This is substantially better than the previous case in which the
number of iterations of one of the loops decreased by a constant
for each iteration!


