%
% comp1.tex Complexity section
%
% This is the common section of the complexity section - used in
% both HTML and PS versions
%
% Fast inline math

\setcounter{section}{4}
\section{Complexity}

We have already used the \Ohx\ notation to 
denote the general behaviour of an algorithm as a 
function of the problem size.
We have said that an algorithm is
\Oh{\log{n}}
if its running time, $T(n)$, to solve a 
problem of size \nx\ is proportional
to \logn.

% \begin{rawhtml}
% <H3>4.1 The O Notation</H3>
% \end{rawhtml}
\subsection{The \Ohx\ notation}

Formally, 
$O(g(n))$ is the {\em set}
of functions, $f$, such that
for some $c > 0$,
\[ f(n) < c g(n) \]
for all positive integers, $n > N$,
ie for all sufficiently large $N$.
Another way of writing this is:
\[ \displaystyle{ \lim_{n \rightarrow \infty}} \frac{f(n)}{g(n)} \leq c \]

Informally, we say the $O(g)$ is the set of
all functions which grow {\it no faster} than $g$. 
The function $g$ is an {\em upper bound} to functions in $O(g)$.

We are interested in the set of functions defined by the $O$
notation because we want to argue about the relative merits of
algorithms - independent of their implementations.
That is, we are not concerned with the language or machine used;
we want a means of comparing algorithms which is relevant to 
{\it any} implementation.

We can define two other functions: $\Omega(g)$
and $\Theta(g)$ .

$\Omega(g)$ the set of functions $f(n)$ for which
$f(n) \geq c g(n)$ for all positive integers, $n > N$, and

\[ \Theta(g) = \Omega(g) \cap O(g) \]

We can derive:

\[ f \in \Theta(g) \]

if

\[ \displaystyle{ \lim_{n \rightarrow \infty}} \frac{f(n)}{g(n)} = c \]

Thus, $\Omega(g)$ is a lower bound - functions in $\Omega(g)$ grow
{\it faster} than $g$ and
$\Theta(g)$ are functions that grow {\it at the same rate} as $g$.
In these last two statements - as in most of the discussion on
complexity theory - "within a constant factor" is understood.
Different languages, compilers, machines, operating systems, etc
will produce different constant factors: it is the general behaviour
of the running time as $n$ increases to very large values that
we're concerned with.


