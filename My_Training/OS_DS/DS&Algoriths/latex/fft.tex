
\documentstyle{article}
\begin{document}
\input{../LaTeX/illustrations.tex}
%
% Either indent all the paragraphs or none!
%
\parindent 0in
%
% hanging indented paragraphs
%
\newlength{\indenthang}
\newenvironment{hanging}[1]{
\setlength{\indenthang}{#1}
\begin{list}{xx}{ \itemindent=0pt
\labelwidth=\indenthang
\setlength{\leftmargin}{\labelwidth}
\addtolength{\leftmargin}{4pt}
}}{
\end{list}
}
%
% hangout command used for the 'item' command of list environments
%
\newcommand{\hangout}[1]{\item[#1 \hfill]}

%
% assembly program text environment
%
\newlength{\instab}
\setlength{\instab}{20pt}
\newlength{\optab}
\setlength{\optab}{40pt}
\newlength{\comtab}
\setlength{\comtab}{60pt}
\newenvironment{program}{
\tt
\begin{tabbing}
\hspace{\instab}\=\hspace{\optab}\=\hspace{\comtab}\=\kill
}{\end{tabbing}}
%
% C code text environment
%
\newenvironment{Ccode}{
\addtolength{\leftmargin}{2cm}
% Stop breaking code blocks over page boundaries
\begin{minipage}{20cm}
\begin{tabbing}
\hspace{20pt}\=\hspace{20pt}\=\hspace{20pt}\=\hspace{20pt}\=\kill
\tt
}
{\end{tabbing}
\end{minipage}
}

%
% miscellaneous
%
\newcommand{\IE}{{\it i.e.\ }}
\newcommand{\EG}{{\it e.g.\ }}
\newcommand{\CF}{{\it cf.\ }}
\newcommand{\ETC}{{\it etc.}}
\newcommand{\ETAL}{{\it et al.}}

\newcommand{\hex}[1]{{{\tt #1}$_{16}$}}
\newcommand{\dec}[1]{{{\tt #1}$_{10}$}}
\newcommand{\million}{$\times10^6$}

%
% time
%
\def\clocktime{{\newcount\scratch
 \scratch=\time
 \divide\scratch by 60
 \number\scratch :\multiply\scratch by -60
 \advance\scratch by\time
 \ifnum\scratch<10 0\fi
 \number\scratch}}
% use like so:
%  \date{\today at \clocktime}
%
%
\newcommand{\Ohx}{{\bf O}}
\newcommand{\nx}{{{\it {\bf n}}}}
\newcommand{\varx}[1]{{{\it {\bf #1}}}}

% Often used math
\newcommand{\Oh}[1]{{{\bf O}$(#1)$}}
\newcommand{\logn}{{$\log{n}$}}
%
% Things we've got to get right later
%
\newcommand{\xx}{{\bf ?xx?}}
\newcommand{\yy}{{\bf ?yy?}}
\newcommand{\zz}{{\bf ?zz?}}

\newcommand{\HR}{\rule{12cm}{0.5mm}}
%

\section*{Data Structures and Algorithms}
\HR
\setcounter{section}{11}
\section{Multiplication of Polynomials}
\label{sec:poly-mult}
If we have two polynomials:

\[A(x) = \sum_{j=0}^{n-1}a_{j}x^{j}   \]
and
\[B(x) = \sum_{j=0}^{n-1}b_{j}x^{j}   \]
then, we can form a product:
\[C(x) = A(x)B(x) = \sum_{0}^{2n-2}c_{j}x^{j} \]
where
\[c_{j} = \sum_{k=0}^{j}a_{k}b_{j-k} \]
Clearly, this algorithm is $\Theta(n^2)$.

\newcommand{\PVR}{{Point-Value Representation}}
\subsection{\PVR}

In section \ref{sec:poly-mult}, we used the
{\it coefficient form} to represent the polynomials.
There is an alternative way to represent
polynomials, using the {\it point-value representation}.
A polynomial of degree $n-1$ can be represented by
$n$ point-value pairs:

\[ \{ (x_0,y_0), (x_1,y_1), \ldots , (x_{n-1},y_{n-1}) \} \]

where
\begin{itemize}
\item All the $x_k$ are distinct
\item $y_k = A(x_k)$
\end{itemize}

\subsection{Uniqueness Theorem}

For any set of $n$ points, there is a unique polynomial.

(For a proof, {\it see} Cormen \ETAL).

\subsection{Operations in the \PVR} 

If we have two polynomials, $A$ and $B$:

\[ A(x) = \{ (x_0,y_0), (x_1,y_1), \ldots , (x_{n-1},y_{n-1}) \} \]
and
\[ B(x) = \{ (x_0,y_0'), (x_1,y_1'), \ldots , (x_{n-1},y_{n-1}') \} \]
where all the $x_k$ are the same,
then

\[ C(x) = A(x)+ B(x) = \{ (x_0,y_0+y_0'), (x_1,y_1+y_1'), \ldots , (x_{n-1},y_{n-1}+y_{n-1}') \} \]
and
\[ C(x) = A(x)\dot B(x) = \{ (x_0,y_0y_0'), (x_1,y_1y_1'), \ldots , (x_{n-1},y_{n-1}y_{n-1}') \} \]
giving a $\Theta(n)$ procedure for multiplying two polynomials.
But note that we now need $2n$ points in our representations of $A$
and $B$.

Unfortunately, conversion between coefficient and point-value
representations requires generating $n$ or $2n$ points.
Evaluating $y_k = A(x_k)$ requires $\Theta(n)$ operations 
using Horner's rule,
so the conversion is $\Theta(n^2)$!

However, note that we can choose any $x_k$ as long as they are 
distinct.
If we choose the {\em $n$ complex $n^{th}$ roots of unity},
then we are performing a Discrete Fourier Transform.
As we will show, this takes $\Theta(n\log{n})$ time.
The reverse transformation 
(from point-value to coefficient representation),
called {\it interpolation}, is also a Fourier Transform,
also requiring $\Theta(n\log{n})$ time.

This leads us to the following polynomial multiplication
algorithm:

\begin{tabular}{|p{9cm}|c|}
\hline
Augment polynomials $A$ and $B$
by adding $n$ higher order 0 coefficients & $\Theta(n)$ \\
\hline
{\em Evaluate}

Compute point-value representation of $A(x)$ and $B(x)$
by two applications of the FFT

$\Rightarrow$ values at each $(2n)^{th}$ root of unity. & $\Theta(n\log{n})$ \\
\hline

Point-wise multiply & $\Theta(n)$ \\
\hline

{\em Interpolate}

Create coefficient representation by application of FFT & $\Theta(n\log{n})$ \\
\hline
Total & $\Theta(n\log{n})$ \\
\hline
\end{tabular}

\subsection{Complex Roots of Unity}

The complex $n^{th}$ root of unity, $\omega$, is defined by:
\[ \omega^n = 1 \]
There are exactly $n$ such roots:
\[ e^{\frac{2\pi{ik}}{n}}  
= \cos{\frac{2\pi k}{n}} + i \sin{\frac{2\pi k}{n}}
\ for\ k = 0,1, \ldots ,n-1 
 \]
The {\em principal $n^{th}$ root of
unity} is
\[ \omega_n = e^{\frac{2\pi i}{n}} \]
The other roots are
\[ \omega_n^2, \omega_n^3, \ldots , \omega_n^{n-1} \]

\subsection{Properties of $\omega$}
\begin{tabular}{p{3cm}p{6cm}}
\begin{center}Cancellation \\ lemma
\end{center} &
For any $n\geq 0, k \geq 0, d > 0$:
\[ \omega_{dn}^{dk} = \omega_n^k \]
 \\
 & For any even $n > 0$:
\[ \omega_n^{\frac{n}{2}} = \omega_2 = -1 \] \\
\begin{center}Halving \\ lemma
\end{center} &
If $n$ is even, the squares of the $n$ complex $n^{th}$
roots of unity are the $\frac{n}{2}$ complex $(\frac{n}{2})^{th}$ roots. \\
\end{tabular}

\subsection{The Fast Fourier Transform}


To evaluate $A(x)$, we divide it into two parts:
\[ A^{even}(x) = a_0 + a_2x + a_4x^2 + \ldots + a_{n-2}x^{\frac{n}{2}-1} \]
and
\[ A^{odd}(x) = a_1 + a_3x + a_5x^2 + \ldots + a_{n-1}x^{\frac{n}{2}-1} \]
now
\[ A(x) = A^{even}(x^2) + A^{odd}(x^2)x \]
So evaluating $A(x)$ at
\[ \omega_n^0, \omega_n^1, \omega_n^2, \ldots, \omega_n^{n-1} \]
reduces to 
\begin{enumerate}
\item Evaluating $A^{even}(x)$ and $A^{odd}(x)$ at
\[ (\omega_n^0)^2, (\omega_n^1)^2, (\omega_n^2)^2, \ldots, (\omega_n^{n-1})^2 \]{\em but} these are the complex $(\frac{n}{2})^{th}$ roots of unity
and there are only $\frac{n}{2}$ of them.
\item Combining them.
\end{enumerate}

So we have divided our $\Theta(n^2)$ evaluation into
two sub-problems of size $\frac{n}{2}$.
As usual, we can divide the problem $\log{n}$ times,
giving a  $\Theta(n\log{n})$ evaluation algorithm.

Thus the time complexity of FFT is $\Theta(n\log{n})$ and
we have developed a
$\Theta(n\log{n})$ algorithm for multiplying polynomials.

\subsection{Recursive FFT}

A recursive FFT algorithm has the following form:
\[ 
\begin{array}{l}
fft(a, n) \\
\ \ \ if\ n=1\ then\ return\ a; \\
\ \ \ \omega_n = e^{\frac{2\pi i}{n}} \\
\ \ \ \omega = 1 \\
\ \ \ a^{e} = ( a_0,a_2,a_4,\ldots,a_{n-2}) \\
\ \ \ a^{o} = ( a_1,a_3,a_5,\ldots,a_{n-1}) \\
\ \ \ y^{e} = fft( a^e, \frac{n}{2} ) \\
\ \ \ y^{o} = fft( a^o, \frac{n}{2} ) \\
\ \ \ for\ k = 0\ to\ \frac{n}{2}-1 \\
\ \ \ \ \ \ y_k = y_k^e + \omega y_k^o \\
\ \ \ \ \ \ y_{k+\frac{n}{2}} = y_k^e - \omega y_k^o \\
\ \ \ \ \ \ \omega = \omega \omega_n \\
\ \ \ return\ y
\end{array} \]

\subsection{The Butterfly}

To save the overhead of recursive calls,
efficient FFT routines are iterative,
replacing the $n$ values input to each of the
$\log{n}$ stages with the new values calculated
in that stage.

The combination stage in the recursive algorithm may be
represented diagrammatically:

\begin{center}
\leavevmode
\epsfxsize=8cm
\epsfbox{butterfly.eps}
\end{center}

Standard iterative algorithms simply stack $n$ of these
`butterflies' (so-called because of the shape of the
diagram) vertically
and then replicate this stack for $\log{n}$ stages.
The results from each stage overwrite those from the
previous one. 
A `bit-twiddling' function determines which butterfly in
the stage $k+1$ the outputs of the butterfly in
stage $k$ are sent.



\HR

{\small { \copyright John Morris, 1996 } }
\end{document}
